\textbf{Team contract}

The team consists of the following four members:
\begin{itemize}
\item{Jacob Fisher (jaco@itu.dk)}
\item{Christian Lyngbye (clyn@itu.dk)}
\item{Ivaylo Sharkov (isha@itu.dk)}
\item{Rasmus Greve (ragr@itu.dk)}
\end{itemize}
Our expectations of the project is to create a working product with a report meeting all requirements at the right deadlines. We expect to get at least a passing grade and strive towards getting a 10 or 12 at the exam.

We agree on doing project work at least once a week, usually on fridays from 10am to around 2-4pm depending on the amount of work to be done and the deadlines.

Three group members have an exam in week 42. After this, we should be able to meet some Thursdays as well.

Rasmus is having a baby in the end of October and will not be able to do group work on site at ITU in a couple of weeks after this time. He will do work from home and keep in touch with the group via email and instant messaging.

\textbf{Week 37 (13/09/2013)}

Today we created our course log and a git repository.
Jacob retrieved information for our server slice.

We decided on which technologies we were going to use:
We would use the programming language Java and create a Web Application that would use JSF and ORM to map objects to a relational database.

We chose to user a Tomcat server as the application server.

For source control management we have shared a private repository on Github.

We made some agreements in the group regarding collaborative tools.
We chose to use LaTeX to write the report and hosted it on writeLaTeX.
Google Drive for file sharing.
Facebook for communicating outside of the university.

On the server slice we set up a firewall to filter all ports except 6666 which we use for SSH.
We disallowed root access to SSH.
In order to run the application on the server we downloaded and installed: Java JDK, Apache Tomcat and Git.
We discussed how to deploy to the server and considered building the application locally on the server.

We made an initial draft of user stories and requirements for our system.


\textbf{Week 38 (20/09/2013)}

As an experiment Christian installed PostgreSQL on the server slice and it was proved to work so we chose to use it. 
Christian scanned all team servers with nmap and found some differences in solutions e.g. turning ping responses off and closed ports.
Added an example Netbeans project with JSF and started working on connecting to the database.
Rasmus designed the database diagram created an initial creation script for the database.

Ivaylo analyzed the risks.

Planned to do code review and architectural risk analysis.


\textbf{Week 39 (27/09/2013)}

This week we began work on the project report by determining which chapters we need. Jacob started writing the choices of technology chapter and the introductory chapters.

The goal and risk analysis was elaborated on and importet into the report LaTeX document.

We discussed and planned a social engineering attack targeted at all the other groups involving impersonating the TA in an email. We agreed that it was still too early in the process to perform this attack since most teams would be unlikely to have a working webservice to which we could gain access.

\textbf{Week 40 (04/10/2013)}

Ivaylo was sick today and was unable to do group work.

Christian made some refactoring in the codebase requiring the rest of the team to reimport the project into NetBeans.
This refactoring included renaming the project and introducing the Selenide testing framework and a few sample test cases to be elaborated on later. The system is now called ``FriendSpace''.

Rasmus spent a bit of time styling the webpages using Twitter Bootstrap. Jacob drew a logo to make the visual expression of the webpages more appealing.

The system was deployed to the webserver on port 80 which was opened. Accessing the root of the server redirects the user to /ssase13.

Furthermore, the team spent a lot of time throughout the day finishing up the first draft of the report for the first hand-in.