

\textbf{Week 37 (13/09/2013)}
Today we created our course log and a git repository.
Jacob retrieved information for our server slice.

We decided on which technologies we were going to use:
We would use the programming language Java and create a Web Application that would use JSF and ORM to map objects to a relational database.

We chose to user a Tomcat server as the application server.

For source control management we have shared a private repository on Github.

We made some agreements in the group regarding collaborative tools.
We chose to use LaTeX to write the report and hosted it on writeLaTeX.
Google Drive for file sharing.
Facebook for communicating outside of the university.

On the server slice we set up a firewall to filter all ports except 6666 which we use for SSH.
We disallowed root access to SSH.
In order to run the application on the server we downloaded and installed: Java JDK, Apache Tomcat and Git.
We discussed how to deploy to the server and considered building the application locally on the server.

We made an initial draft of user stories and requirements for our system.


\textbf{Week 38 (20/09/2013)}

As an experiment Christian installed PostgreSQL on the server slice and it was proved to work so we chose to use it. 
Christian scanned all team servers with nmap and found some differences in solutions e.g. turning ping responses off and closed ports.
Added an example Netbeans project with JSF and started working on connecting to the database.
Rasmus designed the database diagram created an initial creation script for the database.

Ivaylo analyzed the risks.


Planned to do code review and architectural risk analysis.

